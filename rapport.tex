\documentclass[11pt,a4paper]{article}
\usepackage[utf8]{inputenc}
\usepackage[T1]{fontenc}
\usepackage[french]{babel}
\usepackage{lmodern}
\usepackage{microtype}
\usepackage{amsmath,amssymb}
\usepackage{longtable}
\usepackage{float}
\usepackage{booktabs}
\usepackage{graphicx}
\usepackage{tikz}
\usepackage{hyperref}
\usepackage{geometry}
\usepackage{makecell}
\usepackage{pdflscape} % <-- for landscape tables
\geometry{margin=2.3cm}

\title{INFO-F310 --- Projet TSP : comparaison MTZ et DFJ}
\author{Anass L'marzguioui 000537609 \and Islam Mekhiouba 000538266}
\date{17/12/2025}

\begin{document}
	\maketitle
	
	\section{Introduction}
	Le problème du voyageur de commerce (TSP) consiste à trouver un cycle hamiltonien de coût minimal visitant chaque ville exactement une fois et revenant au point de départ.
	Nous comparons deux formulations ILP : MTZ et DFJ, ainsi qu'une version DFJ avec génération itérative de contraintes.
	
	\section{Formulations}
	\subsection{Variables et contraintes communes}
	Variables $x_{ij}\in\{0,1\}$ pour $i\neq j$.
	Contraintes de degré :
	\[
	\sum_{j\neq i} x_{ij} = 1,\qquad \sum_{j\neq i} x_{ji} = 1 \qquad \forall i.
	\]
	Objectif :
	\[
	\min \sum_{i\neq j} c_{ij} x_{ij}.
	\]
	
	\subsection{MTZ}
	Variables auxiliaires $u_i$ pour $i\in\{1,\dots,n-1\}$ avec $1\le u_i\le n-1$.
	Contraintes MTZ :
	\[
	u_i - u_j + (n-1)x_{ij} \le n-2 \qquad \forall i\neq j,\ i,j\in\{1,\dots,n-1\}.
	\]
	Si $x_{ij}=1$, alors $u_j\ge u_i+1$ : un sous-tour induit une contradiction.
	
	\subsection{DFJ}
	Pour tout sous-ensemble $S\subset V$ avec $2\le |S|\le n-1$ :
	\[
	\sum_{i\in S}\sum_{j\in S,\ j\neq i} x_{ij} \le |S|-1.
	\]
	Génération itérative : résoudre sans coupes, détecter les cycles, ajouter une coupe DFJ par sous-tour, répéter.
	
	\section{Tâche 2 : exemple à 5 villes}
	Deux sous-tours :
	\[
	0\to 1\to 0 \qquad\text{et}\qquad 2\to 3\to 4\to 2.
	\]
	\begin{figure}[htbp]
		\centering
		\begin{tikzpicture}[scale=1, every node/.style={circle,draw,minimum size=8mm}]
			\node (0) at (0,0) {0};
			\node (1) at (2,0) {1};
			\node (2) at (0,-2) {2};
			\node (3) at (2,-2) {3};
			\node (4) at (1,-3.5) {4};
			\draw[->,thick] (0) -- (1);
			\draw[->,thick] (1) -- (0);
			\draw[->,thick] (2) -- (3);
			\draw[->,thick] (3) -- (4);
			\draw[->,thick] (4) -- (2);
		\end{tikzpicture}
		\caption{Deux sous-tours sur 5 villes.}
	\end{figure}
	
	\paragraph{Violation MTZ.}
	Dans le cycle $(2,3,4)$, $x_{23}=x_{34}=x_{42}=1$ implique :
	$u_3\ge u_2+1$, $u_4\ge u_3+1$, $u_2\ge u_4+1$ donc contradiction.
	
	\paragraph{Violation DFJ.}
	Pour $S=\{2,3,4\}$, la somme des arcs internes vaut 3, mais DFJ impose $\le |S|-1=2$.
	
	\section{Tâche 3 : résultats expérimentaux}
	\subsection{Méthodologie}
	Solveur : CBC via PuLP. Temps mesuré : uniquement \texttt{prob.solve()}.
	DFJ énumératif : uniquement pour $n\le 15$.
	
	\subsection{Résultats par instance}
	
	% ---- Wide table: put it in landscape to remove the huge Overfull \hbox warnings ----
	\begin{landscape}
		\scriptsize
		\setlength{\tabcolsep}{3pt}
		\renewcommand{\arraystretch}{2.0}
		\setlength{\LTleft}{0pt}
		\setlength{\LTright}{0pt}
		
		\begin{longtable}{@{}p{7cm}r r r r r r r r r r r r r r@{}}
			\caption{Comparaison des formulations (entier) : objectif, temps solveur, tailles de modèles.}\\
			\toprule
			Instance & n &
			\makecell{Obj\\MTZ} & \makecell{T\\MTZ} & \makecell{Vars\\MTZ} & \makecell{Constr\\MTZ} &
			\makecell{Obj\\DFJ\_enum} & \makecell{T\\DFJ\_enum} & \makecell{Vars\\DFJ\_enum} & \makecell{Constr\\DFJ\_enum} &
			\makecell{Obj\\DFJ\_iter} & \makecell{T\\DFJ\_iter} & \makecell{Vars\\DFJ\_iter} & \makecell{Constr\\DFJ\_iter} & \makecell{Iter\\DFJ\_iter} \\
			\midrule
			\endfirsthead
			
			\toprule
			Instance & n &
			\makecell{Obj\\MTZ} & \makecell{T\\MTZ} & \makecell{Vars\\MTZ} & \makecell{Constr\\MTZ} &
			\makecell{Obj\\DFJ\_enum} & \makecell{T\\DFJ\_enum} & \makecell{Vars\\DFJ\_enum} & \makecell{Constr\\DFJ\_enum} &
			\makecell{Obj\\DFJ\_iter} & \makecell{T\\DFJ\_iter} & \makecell{Vars\\DFJ\_iter} & \makecell{Constr\\DFJ\_iter} & \makecell{Iter\\DFJ\_iter} \\
			\midrule
			\endhead
			
			\bottomrule
			\endfoot
			
			10\_circle1 & 10 & 298.57 & 1.939 & 99 & 92 & 298.57 & 0.503 & 90 & 1032 & 298.57 & 0.274 & 90 & 28 & 2 \\
			10\_euclidean1 & 10 & 207.85 & 0.263 & 99 & 92 & 207.85 & 0.222 & 90 & 1032 & 207.85 & 0.249 & 90 & 25 & 1 \\
			10\_euclidean2 & 10 & 323.92 & 0.304 & 99 & 92 & 323.92 & 0.202 & 90 & 1032 & 323.92 & 0.276 & 90 & 28 & 2 \\
			10\_line1 & 10 & 156.14 & 1.603 & 99 & 92 & 156.14 & 0.222 & 90 & 1032 & 156.14 & 0.413 & 90 & 33 & 4 \\
			10\_random\_asym1 & 10 & 191.53 & 0.110 & 99 & 92 & 191.53 & 0.208 & 90 & 1032 & 191.53 & 0.085 & 90 & 20 & 0 \\
			10\_random\_sym1 & 10 & 280.99 & 0.442 & 99 & 92 & 280.99 & 0.218 & 90 & 1032 & 280.99 & 0.266 & 90 & 28 & 2 \\
			12\_circle1 & 12 & 265.22 & 1.324 & 143 & 134 & 265.22 & 2.450 & 132 & 4106 & 265.22 & 0.597 & 132 & 39 & 4 \\
			12\_line1 & 12 & 174.80 & 5.279 & 143 & 134 & 174.80 & 1.201 & 132 & 4106 & 174.80 & 0.865 & 132 & 43 & 6 \\
			14\_circle1 & 14 & 304.48 & 2.713 & 195 & 184 & 304.48 & 4.859 & 182 & 16396 & 304.48 & 0.242 & 182 & 34 & 1 \\
			14\_line1 & 14 & 188.02 & 20.775 & 195 & 184 & 188.02 & 5.535 & 182 & 16396 & 188.02 & 0.794 & 182 & 53 & 7 \\
			15\_euclidean1 & 15 & 306.70 & 4.105 & 224 & 212 & 306.70 & 10.487 & 210 & 32781 & 306.70 & 0.880 & 210 & 50 & 5 \\
			15\_euclidean2 & 15 & 345.91 & 10.825 & 224 & 212 & 345.91 & 11.162 & 210 & 32781 & 345.91 & 0.405 & 210 & 45 & 3 \\
			15\_random\_asym1 & 15 & 342.94 & 0.238 & 224 & 212 & 342.94 & 10.298 & 210 & 32781 & 342.94 & 0.219 & 210 & 32 & 1 \\
			15\_random\_sym1 & 15 & 377.50 & 1.140 & 224 & 212 & 377.50 & 10.143 & 210 & 32781 & 377.50 & 0.191 & 210 & 37 & 1 \\
			20\_euclidean1 & 20 & 370.13 & 0.912 & 399 & 382 &  &  &  &  & 370.13 & 0.273 & 380 & 50 & 1 \\
			20\_euclidean2 & 20 & 396.75 & 23.903 & 399 & 382 &  &  &  &  & 396.75 & 0.343 & 380 & 56 & 2 \\
			20\_random\_asym1 & 20 & 324.08 & 0.507 & 399 & 382 &  &  &  &  & 324.08 & 0.242 & 380 & 43 & 1 \\
			20\_random\_sym1 & 20 & 371.10 & 4.267 & 399 & 382 &  &  &  &  & 371.10 & 1.357 & 380 & 59 & 5 \\
			25\_euclidean1 & 25 & 410.26 & 12.907 & 624 & 602 &  &  &  &  & 410.26 & 1.696 & 600 & 72 & 4 \\
			25\_euclidean2 & 25 & 418.56 & 44.769 & 624 & 602 &  &  &  &  & 418.56 & 0.628 & 600 & 74 & 4 \\
		\end{longtable}
		
		\normalsize
	\end{landscape}
	
	
	\subsection{Synthèse (moyennes par taille $n$)}
	\small
	\begin{tabular}{rlllllll}
		\toprule
		n & Formulation & Temps moyen (s) & Temps médian (s) & Constr moy. & Vars moy. & Gap moy. & Iter moy. \\
		\midrule
		10 & DFJ\_enum & 0.263 & 0.220 & 1032.0 & 90.0 & 0.000 &  \\
		10 & DFJ\_iter & 0.260 & 0.270 & 27.0 & 90.0 &  & 1.83 \\
		10 & MTZ & 0.777 & 0.373 & 92.0 & 99.0 & 0.129 &  \\
		12 & DFJ\_enum & 1.825 & 1.825 & 4106.0 & 132.0 & -0.000 &  \\
		12 & DFJ\_iter & 0.731 & 0.731 & 41.0 & 132.0 &  & 5.00 \\
		12 & MTZ & 3.301 & 3.301 & 134.0 & 143.0 & 0.347 &  \\
		14 & DFJ\_enum & 5.197 & 5.197 & 16396.0 & 182.0 & -0.000 &  \\
		14 & DFJ\_iter & 0.518 & 0.518 & 43.5 & 182.0 &  & 4.00 \\
		14 & MTZ & 11.744 & 11.744 & 184.0 & 195.0 & 0.444 &  \\
		15 & DFJ\_enum & 10.523 & 10.393 & 32781.0 & 210.0 & 0.000 &  \\
		15 & DFJ\_iter & 0.424 & 0.312 & 41.0 & 210.0 &  & 2.50 \\
		15 & MTZ & 4.077 & 2.623 & 212.0 & 224.0 & 0.167 &  \\
		20 & DFJ\_iter & 0.554 & 0.308 & 52.0 & 380.0 &  & 2.25 \\
		20 & MTZ & 7.397 & 2.590 & 382.0 & 399.0 & 0.087 &  \\
		25 & DFJ\_iter & 1.162 & 1.162 & 73.0 & 600.0 &  & 4.00 \\
		25 & MTZ & 28.838 & 28.838 & 602.0 & 624.0 & 0.141 &  \\
		\bottomrule
	\end{tabular}
	\normalsize
	
	\paragraph{Analyse.}
	DFJ\_iter est en général le plus rapide : il conserve une relaxation serrée et n'ajoute que quelques contraintes utiles.
	MTZ est plus compact mais souvent pénalisé par une relaxation moins serrée.
	
	\section{Tâche 4 : relaxation continue et integrality gap}
	\[
	\text{gap} = \frac{z_{int}-z_{relax}}{z_{int}}.
	\]
	\subsection{Gaps par instance}
	\small
	\begin{longtable}{@{}p{4.2cm}rrp{2.0cm}rr@{}}
		\toprule
		Instance & n & Type & Gap(MTZ) & Gap(DFJ\_enum) \\
		\midrule
		\endfirsthead
		\toprule
		Instance & n & Type & Gap(MTZ) & Gap(DFJ\_enum) \\
		\midrule
		\endhead
		\bottomrule
		\endfoot
		instance\_10\_circle\_1.txt & 10 & circle & 0.163 & 0.000 \\
		instance\_10\_euclidean\_1.txt & 10 & euclidean & 0.036 & 0.000 \\
		instance\_10\_euclidean\_2.txt & 10 & euclidean & 0.034 & 0.000 \\
		instance\_10\_line\_1.txt & 10 & line & 0.477 & 0.000 \\
		instance\_10\_random\_asym\_1.txt & 10 & random\_asym & 0.000 & 0.000 \\
		instance\_10\_random\_sym\_1.txt & 10 & random\_sym & 0.066 & 0.000 \\
		instance\_12\_circle\_1.txt & 12 & circle & 0.332 & 0.000 \\
		instance\_12\_line\_1.txt & 12 & line & 0.362 & -0.000 \\
		instance\_14\_circle\_1.txt & 14 & circle & 0.269 & 0.000 \\
		instance\_14\_line\_1.txt & 14 & line & 0.619 & -0.000 \\
		instance\_15\_euclidean\_1.txt & 15 & euclidean & 0.253 & 0.000 \\
		instance\_15\_euclidean\_2.txt & 15 & euclidean & 0.295 & 0.000 \\
		instance\_15\_random\_asym\_1.txt & 15 & random\_asym & 0.001 & 0.000 \\
		instance\_15\_random\_sym\_1.txt & 15 & random\_sym & 0.118 & 0.000 \\
		instance\_20\_euclidean\_1.txt & 20 & euclidean & 0.146 &  \\
		instance\_20\_euclidean\_2.txt & 20 & euclidean & 0.151 &  \\
		instance\_20\_random\_asym\_1.txt & 20 & random\_asym & 0.014 &  \\
		instance\_20\_random\_sym\_1.txt & 20 & random\_sym & 0.039 &  \\
		instance\_25\_euclidean\_1.txt & 25 & euclidean & 0.154 &  \\
		instance\_25\_euclidean\_2.txt & 25 & euclidean & 0.129 &  \\
	\end{longtable}
	\normalsize
	
	\section{Figures}
	\begin{figure}[H]
		\centering
		\includegraphics[width=0.8\linewidth]{figures/time_vs_n.png}
		\caption{Temps solveur (entier) moyen en fonction de \texorpdfstring{$n$}{n}.}
	\end{figure}
	
	\begin{figure}[H]
		\centering
		\includegraphics[width=0.8\linewidth]{figures/constr_vs_n.png}
		\caption{Nombre moyen de contraintes en fonction de \texorpdfstring{$n$}{n}.}
	\end{figure}
	
	\begin{figure}[H]
		\centering
		\includegraphics[width=0.9\linewidth]{figures/gap_vs_n.png}
		\caption{Integrality gap moyen (MTZ vs DFJ\_enum) en fonction de \texorpdfstring{$n$}{n}.}
	\end{figure}
	
	\begin{figure}[H]
		\centering
		\includegraphics[width=0.9\linewidth]{figures/iters_vs_n.png}
		\caption{DFJ itératif : itérations moyennes en fonction de \texorpdfstring{$n$}{n}.}
	\end{figure}
	
	\section{Tâche 5 : questions de réflexion}
	\subsection{Pourquoi énumérer les tours est impraticable, mais DFJ est praticable ?}
	Énumérer explicitement toutes les tournées implique un nombre factoriel $((n-1)!/2)$.
	DFJ a un nombre exponentiel théorique de contraintes, mais la génération itérative n'ajoute que les contraintes violées (sous-tours rencontrés), ce qui suffit en pratique.
	
	\subsection{Pourquoi DFJ peut être plus rapide que MTZ malgré plus de contraintes ?}
	DFJ a une relaxation plus serrée : le solveur explore moins de nœuds en branch-and-bound.
	MTZ, plus compact, peut produire des relaxations plus lâches, augmentant l'exploration.
	
	\subsection{Bonus : 2 sous-tours}
	Si exactement deux cycles $S$ et $\bar S$ existent, ajouter une seule contrainte DFJ sur l'un des deux suffit (l'autre devient redondante car complémentaire).
	On peut vérifier en n'ajoutant qu'une coupe dans ce cas et en comparant temps/itérations.
	
	\section{Conclusion}
	DFJ itératif offre le meilleur compromis pratique : relaxation forte et peu de contraintes ajoutées.
	MTZ reste simple mais peut souffrir d'un gap plus élevé sur certaines instances.
	
\end{document}
